% -*- UTF-8 : LaTeX -*-
%%
%% 2013/09/28. xoblivoir ver2 버전으로 포팅.
%% 2010/03/17. xoblivoir 버전으로 포팅
%%
%% memhangul-ucs v1.44 테스트 파일
%% Written by 도은이아빠.
%%

\documentclass[demo,chapter,openany,amsmath,gremph,adjustmath]{xoblivoir}
%%% amsmath 옵션은 amsmath, amssymb 패키지를 로드해준다..

%\usepackage[logonly]{trace} %% for debugging.
\let\traceon\relax\let\traceoff\relax

%%% display overfullrule
%\setlength\overfullrule{5pt}

%% 페이지 레이아웃. fapapersize를 이용한다..
\usepackage{fapapersize}
\usefapapersize{220mm,276mm,20mm,80mm,30mm,35mm}

%% 폰트 설정
\def\DefaultFont{HCR Batang LVT}
\ifPDFTeX
	\usepackage{mathpazo}
\else\ifLuaOrXeTeX
	\defaultfontfeatures{Mapping=tex-text,Ligatures=TeX}
	\setmainfont{TeX Gyre Pagella}
	\usepackage{unicode-math}
	\setmathfont{Asana-Math.otf}
 	\setsansfont[Scale=.95]{TeX Gyre Heros}
 	\setkormainfont[ItalicFont=\DefaultFont,ItalicFeatures={FakeSlant=.167}]{\DefaultFont}[]{\DefaultFont}
 	\setkorsansfont[]{NanumGothic}[]{HCR Dotum LVT}
\fi\fi

\ifLuaTeX
\def\interHANGUL{InterHangul}
\let\circemph=\dotemph
\else\ifXeTeX
\def\interHANGUL{interhchar}
\fi\fi

%% logos
\usepackage{hologo}

%%% marginfix
%\usepackage{marginfix}

%% 좌우 마진의 marginpar 위치가 혼선을 보인다면
%% 다음을 선언한다.
\strictpagecheck

\usepackage{cite}

\ifPDFTeX
\usepackage{graphicx}
\fi
\usepackage[dvipsnames]{xcolor}
%% pdf 정보
\hypersetup{%
    bookmarks=true,%
    plainpages=false,%
    colorlinks=true,%
    pdfauthor={Karnes Kim},%
    pdfcreator={Karnes Kim}
}

\nonfrenchspacing
%% nonfrench를 설정하는 경우에는 \xspaceskip도 정해주자.
%% 여기서는 눈에 띄도록 이 길이를 넉넉하게 잡았음.
%% 일반적으로 .6 내지 .7 정도를 권장함.
\xspaceskip=.8em plus .1em minus .1em

%% 이 아래 명령들은 본문에서 바꿀 수 있다.
%% 행간설정. 두번째 인자는 fn/float에 적용되는 행간.
\SetHangulspace{1.5}{1.1}
%% quotespacing을 설정함. \noadjustqutespacing이 default. \adjustquotespacing
%\adjustquotespacing
%% float/fn의 좁은 행간 설정을 disable. \adjustfloatfnspacing
\noadjustfloatfnspacing

%% snugshade 환경에 칠할 배경색.
\definecolor{shadecolor}{gray}{0.85}

%%% from memman.tex. modified.
%% 본문에서 한 번 사용하기 위해 설정한 chapter
%% style 예제. 이 예제는 memman에 있다.
%% 한글판을 위하여 조금 수정함.
\makeatletter
\newlength{\numberheight}
\newlength{\barlength}
\makechapterstyle{veelo}{%
   \setlength{\beforechapskip}{40pt}
   \setlength{\midchapskip}{25pt}
   \setlength{\afterchapskip}{40pt}
   \renewcommand{\chapnamefont}{\normalfont\LARGE\flushright}
   \renewcommand{\chapnumfont}{\normalfont\HUGE}
   \renewcommand{\chaptitlefont}{\normalfont\HUGE\bfseries\flushright}
   \renewcommand{\printchaptername}{}%
   \renewcommand{\prechapternum}{% <= 이 명령을 정의. 
            \chapnamefont\MakeUppercase{chapter}}
   \renewcommand{\postchapternum}{}% <= 이 명령을 정의. 여기서는 비움.
   \renewcommand{\chapternamenum}{}
   \setlength{\numberheight}{18mm}
   \setlength{\barlength}{\paperwidth}
   \addtolength{\barlength}{-\textwidth}
   \addtolength{\barlength}{-\spinemargin}
   \renewcommand{\printchapternum}{%
      \makebox[0pt][l]{% 
      \hspace{.8em}%
      \resizebox{!}{\numberheight}{\chapnumfont \thechapter}% 
      \hspace{.8em}%
      \rule{\barlength}{\numberheight}
     }
   }
   \makeoddfoot{plain}{}{}{\thepage}
}

\makechapterstyle{mycompanion}{%
  \chapterstyle{default}
  \renewcommand*{\chapnamefont}{\normalfont\LARGE\scshape}
  \renewcommand*{\printchaptername}{\raggedleft\chapnamefont \@chapapp}
  \renewcommand*{\prechapternum}{\raggedleft\chapnamefont \pre@chapter}
  \renewcommand*{\chapnumfont}{\normalfont\Huge}
  \setlength{\chapindent}{\marginparsep}
  \renewcommand*{\printchaptertitle}[1]{%
    \begin{adjustwidth}{}{-\chapindent}
      \raggedleft \chaptitlefont ##1\par\nobreak
    \end{adjustwidth}}} 

\makeatother 

%%% User's commands
%% 사용자 명령들. 인덱스 관련 명령.
\newcommand\dispcmd[1]{%
    \texttt{\textbackslash #1}%
    \index{명령!\textbackslash #1}%
    \index{#1@\textbackslash #1}%
}

\newcommand\cls[1]{%
    \texttt{#1}\ 클래스%
    \index{클래스!#1}%
    \index{#1~클래스}%
}

\newcommand\pkg[1]{%
    \texttt{#1}\ 패키지%
    \index{패키지!#1}%
    \index{#1~패키지}%
}

\newcommand\thisclass{%
    \texttt{memoir}\ 클래스%
    \index{클래스!memoir}%
    \index{memoir~클래스}%
}

\newcommand\env[1]{%
    \textsf{#1} 환경%
    \index{환경!#1}%
    \index{#1~환경}%
}

\newcommand\wi[2][\empty]{%
    \ifx#1\empty
        \index{#2}#2%
    \else
        \index{#1!#2}#2%
    \fi
}

%% showcommand/showenv 명령.
%% 만약 command/env 보여주기를 끄려면
%% \showcommandfalse를 선언한다.
\makeatletter
\newif\if@showcommand\@showcommandtrue
\newcommand\showcommandtrue{\@showcommandtrue}
\newcommand\showcommandfalse{\@showcommandfalse}

\strictpagechecktrue

\newcommand\showcommand[1]{%
	\if@showcommand
	 \checkoddpage\ifoddpage
      \marginpar{\small\texttt{\textbackslash #1}}%
   \else
      \marginpar{\hfill\small\texttt{\textbackslash #1}}%
   \fi
	\fi
}

\newcommand\showenv[1]{%
	\if@showcommand
	 \checkoddpage\ifoddpage
    \marginpar{\small\textit{#1}(env.)}%
   \else
    \marginpar{\hfill\small\textit{#1}(env.)}%
   \fi
	\fi
}
\makeatother

%% MakeShortVerb
\MakeShortVerb{\|} 
% \DeleteShortVerb{\|}

%% 인덱스의 hyperpage 처리를 위해서
\newcommand\bfhypidx[1]{%
	\textbf{\hyperpage{#1}}%
}

%% replace \bigskip
\newcommand\alineskip{%
	\vspace{\onelineskip}%
}

%% showindexmark
%% 여백을 충분히 확보하고 다음 행을
%% 활성화해볼 것.
%\showindexmarktrue

%% 문장 부호 관련 명령은 스타일에 포함시켰음.
%% \bnm, \snm, \cnm, \ccnm.

%% index를 만든다. 
%% makeindex를 실행할 때는 makeindex-dhucs.pl을 이용할 것.
%% #> perl makeindex-dhucs.pl -s dhhangulucs.ist <filename>
%%
\makeindex

%% newlist
%% 새로운 리스트를 만드는 것이 정말 너무나 간단하다.
\newcommand\queryfont{\raggedleft\sffamily\small}
\newcommand\listofqueriesname{Queries~목록}
\newlistof{listofqueries}{loq}{\listofqueriesname}
\newlistentry{query}{loq}{0}
\newcommand{\query}[2][\empty]{%
	\refstepcounter{query}
	\par\noindent\fbox{Q?}~\begingroup\queryfont #2\endgroup%
	\addcontentsline{loq}{query}{\protect\numberline{\thequery}#2}\par
	\ifx#1\empty\else\index{Query!#1}\fi
}

%% title page
\newcommand\MakeTitle{%
 \begin{titlingpage}
 \setcounter{page}{-1}%
 \begin{adjustwidth*}{0mm}{-55mm}
 \newlength\tmplen\setlength\tmplen{\textwidth}\addtolength\tmplen{60mm}
 \fbox{%
   \begin{minipage}{\tmplen}
   \vspace*{90mm}
   \begin{center}
     \LARGE\bfseries\thetitle \\ \vskip\onelineskip
     \normalfont\normalsize\theauthor
   \end{center}
   \vspace*{100mm}
 \end{minipage}}
 \end{adjustwidth*}
 \end{titlingpage}
}

%% 각주. footmisc는 memoir와 함께 쓸 수 있다.
%\usepackage[bottom]{footmisc}
%% 그러나, 위의 예와 같이 [bottom] footnote를 위해서라면 
%% 다음과 같이 할 것을 권장함. raggedbottom에서 동작함.
\renewcommand*{\footnoterule}{\kern-3pt\vfill
    \hrule width 0.4\columnwidth \kern 2.6pt}
%% 각주를 floats 아래.
\feetbelowfloat

%% 밑줄.
\ifLuaTeX\else
\usepackage[normalem]{ulem}
\fi

%% 사소한 설정
\def\util#1{\texttt{#1}\index{Utility!#1}}

%% \useremph compatibility
\ifPDFTeX
  \let\ORIGuseremph=\useremph
  \def\useremphchar{\tiny★}
  \setlength\useremphraisedim{10pt}
  \def\useremph#1{\ORIGuseremph{#1}}
\else\ifXeTeX
  \let\ORIGuseremph=\useremph
  \def\useremph#1{\ORIGuseremph[10pt]{\tiny★}{#1}}
\else\ifLuaTeX
  \let\ORIGdotemph=\dotemph
  \def\dotemph#1{%
	\def\dotemphraise{0.4em}
	\def\dotemphchar{˙}
	\ORIGdotemph{#1}}	
  \def\circemph#1{%
	\def\dotemphraise{0.4em}
	\def\dotemphchar{˚}
	\ORIGdotemph{#1}}
  \def\useremph#1{%
	\def\dotemphraise{10pt}
	\def\dotemphchar{\tiny★}
	\ORIGdotemph{#1}}
\fi\fi\fi

%%%
%\headnamereftrue

%%%%%%%%%%%%%%%%%%%%%%%%%%%%%%%%%%%%%%%%%%%%%%%%%%%%%%
%%%%%%%%%%%% 이제 문서를 시작.
%%%%%%%%%%%%%%%%%%%%%%%%%%%%%%%%%%%%%%%%%%%%%%%%%%%%%%
\begin{document}

\title{memoir에서 한글을 쓰자!}
\author{도은이아빠}
\date{\today}

%% 아래 \MakeTitle 명령은 이 문서의 preamble에
%% 정의되어 있다. 표지 디자인은 이 정의를 참고하여
%% 수정하여 볼 것.
\MakeTitle
\cleardoublepage

\frontmatter
% ToC, etc
%% 차례 페이지들. 여기서는 demo chapter style과
%% Ruled 페이지 스타일을 적용한다.
\chapterstyle{demo}

\tableofcontents
%\clearpage
\listoffigures
%\clearpage
\listoftables
%% listofqueries는 \newlistof로 만든
%% local \listofx임.
%\clearpage
\listofqueries

\mainmatter

%% hangul 페이지 스타일을 테스트하기
%% 위하여 제1편 앞에 한 챕터를 둠. 다만 chapter의 첫 면은
%% chapter 페이지 스타일을 따른다.
\pagestyle{hangul}

\chapterstyle{default}

\chapter{xoblivoir}

이 문서는 xoblivoir의 몇 가지 기능을 시험하기 위한 목적으로 
작성되었다. 이 문서의 원본은 memhangul을 개발하던 당시
테스트용으로 사용된 바, 가장 먼저 만들어진 oblivoir 문서이다.

해당 문단의 여백에 사용되고 있는 명령이 나와 있으므로 참고할 수 있을
것이다. 실제로 xoblivoir 사용설명서를 작성하기 어려운 것이
기본적인 것은 모두 memoir에 바탕을 두고 있는 데다가
한글화하면서 변화한 점도 많아서 이것을 모두 모으면 너무나 방대한
내용이 된다는 점이다.

이 문서의 원본을 잘 연구하면 그럭저럭 oblivoir가 어떻게 동작하는지
짐작할 수 있게 되기를 희망한다.

이 문서는 다음 순서로 컴파일할 수 있다. |$ENGINE|은 \util{pdflatex}, \util{xelatex}, \util{lualatex} 가운데 하나이고
|$JOBNAME|은 파일 이름(\texttt{xobtesttest})이다.
한편, 색인을 만드는 유틸리티 |$MAKEINDEXENGINE|은 \util{pdflatex} 상황에서는
\util{komkindex}이고 그밖의 경우에는 \util{kotexindy}이다.

\begin{verbatim}
 #  $ENGINE $JOBNAME
 #  $ENGINE $JOBNAME
 #  $MAKEINDEXENGINE $JOBNAME.idx
 #  $ENGINE $JOBNAME
\end{verbatim}



\chapter{default 페이지 스타일}

\section{맹동야를 보내는 글}
\epigraph{한유 글, \cnm{送孟東野序}}{김학주 옮김}

대개 만물은 평정을 얻지 못하면 소리를 내게 된다.
초목은 소리가 없으나 바람이 흔들면 소리를 내게 되며, 물은 소리가 없으나 바람이
움직이면 소리를 내게 된다. 물이 뛰어오르는 것은 바위같은 곳에 부칮쳤기
때문이다. 물이 세차게 끓어오르는 것은 한 곳에서 물결을 막기 때문이다.
물이 펄펄 끓어오르는 것은 불로 태우기 때문이다. 쇠나 돌은 소리가 없으나
치면 소리를 낸다. 사람이 말하는 데 있어서도 이와 같으니,
부득이한 일이 있은 후에야 말을 하게 된다. 노래를 하는 것은 생각이
있기 때문이며, 우는 것은 회포가 있기 때문이다.

음악이라는 것은 가슴속이 막혀 답답할 때 밖으로 새어나오는 것이며
소리를 잘 내는 것을 선택하여 이것을 빌려서 소리를 내게 된다.
쇠\cntrdot 돌\cntrdot 실\cntrdot 대\cntrdot 박\cntrdot 흙\cntrdot 
가죽\cntrdot 나무 등 여덟 가지 악기를 만드는 데 쓰이는 자료들은 만물
가운데 소리를 잘 내는 것들이다.

자연의 계절에 있어서도 역시 그러하니 소리를 잘 내는 것을 선택하여 그것을
빌려서 소리를 내게 된다. 그러므로 새를 빌려 봄의 소리를 내고, 우뢰를
빌려 여름의 소리를 내고, 벌레를 빌어 가을의 소리를 내며,
바람을 빌어 겨울의 소리를 낸다. 사계절이 서로 바뀌어 나타나는 현상은
반드시 그 평정을 얻지 못했기 때문일 것이다.

이는 사람에게 있어서도 마찬가지이다. 사람의 소리 가운데 정묘한 것이 언어이며
문장의 표현은 언어 가운데서도 더욱 정묘한 것이다. 그 중에서도 더욱 소리를
잘 내는 것을 선택하여 이것을 빌려서 소리를 내게 된다.

당요\cntrdot 우순 시대에는 고요와 우가 소리를 잘 내는 사람들이어서
그들을 빌려 소리를 냈다. 기는 문사로써 소리를 내지는 못했으나
스스로 소를 빌려서 소리를 냈다. 하나라 때에는 오자가 노래를 불러 소리를
냈다. 이윤은 은나라에서 소리를 냈고 주공은 주나라에서 소리를 냈다.

무릇 \snm{시}\cntrdot \snm{서} 등 육예에 실린 것들은 모두 소리를
잘 낸 것들이다. 주나라가 쇠퇴해지자 공자의 무리들이 소리를 냈는데
그 소리를 크게 멀리 들렸다. 옛 서적에 ``하늘이 장차 선생을 목탁으로
삼으려 하는구나!''\,라고 하였는데도 믿지 못하겠는가!

주나라 말엽에 이르러서는 장주가 황당한 문사로써 초나라에서
소리를 냈다. 초나라를 큰나라였는데 망할 무렵이 되어 굴원이 소리를 냈다.
장손진\cntrdot 맹가\cntrdot 순경은 도로써 소리를 낸 자들이고,
양주\cntrdot 묵적\cntrdot 관이오\cntrdot 안영\cntrdot 노담\cntrdot 신불해\cntrdot 
한비\cntrdot 신도\cntrdot 전연\cntrdot 추연\cntrdot 시교\cntrdot 손무\cntrdot 
장의\cntrdot 소진의 무리들은 모두 술법으로써 소리를 냈다.

진나라가 융성하자 이사가 소리를 냈으며 한나라 때에는 사마천\cntrdot 사마상여\cntrdot 
양응이 가장 소리를 잘 낸 자들이다.

그 후 위\cntrdot 진 시대에는 소리를 내는 자들이 옛날 사람들에 미치지 못했지만
또한 아직 끊이지는 않았었다. 그 가운데 괜찮은 것들도 그 소리는 맑지만
경박하고 그 음절은 빠르고 급하며 그 문사는 음란하고 슬프며
그 뜻은 느슨하고도 방자하며 그 표현은 난잡하고 문채가 없었으니
하늘이 그 덕을 추하게 여겨 돌보지 않은 때문이었는가? 무엇 때문에
소리를 잘 내는 자들로 하여금 소리를 내게 하지 않았는가!

당(唐)나라가 천하를 장악하고나서는 진자앙\cntrdot 소원명\cntrdot 원결\cntrdot 
이백\cntrdot 두보\cntrdot 이관 등이 모두 자신의 잘하는 것으로써 소리를 내었다.

현재 살아 있으면서 아랫자리에 있는 사람으로 동야 맹교가 비로소 시로써
소리를 내었다. 그는 위\cntrdot 진 시대 사람들보다 훨씬 뛰어나며
게을리하지 않으면 옛사람들의 수준에 미칠 수 있겠고
그밖의 작품들은 한나라의 문풍에 젖어 있다. 나에게서 배운 자들로서
이고와 장적이 가장 뛰어나다. 이 세 사람의 소리는 진실로 훌륭하다.

그런데 하늘이 장차 그들의 소리를 온화하게 하여 국가의 성대함을 소리내게
할 것인지 아니면 장차 그들 자신을 가난하고 굶주리게 하고
그들의 마음을 근심스럽게 하여 그 불행을 스스로 소리내게 할 것인지 모르겠다.
이 세 사람의 운명은 하늘에 달려 있는 것이니 윗자리에 있다고 해서 어찌
기뻐하겠으며 아랫자리에 있다고 해서 어찌 슬퍼하겠는가.

동야가 강남에 근무하러 떠나면서 즐거워하지 않는 것 같아서 내가 그의
운명이 하늘에 달려 있다고 말하며 이를 풀어주려고 하는 것이다.

%% 본문시작.
%% 본문은 companion chapter style로 식자할 것인데,
%% default를 한 번 부른 이유는 앞서 사용한 demo의
%% 글꼴 설정에 영향을 받기 때문이다.

\part{xoblivoir 테스트}

\chapterstyle{mycompanion}
\pagestyle{companion}

%% 제목은 두 줄로 식자하되, heading과 toc에는 개행 없이.
% \chapter[memoir 클래스에서 한글을 쓰자][memoir 클래스에서...]{memoir
%   클래스에서\\ 한글을 쓰자} % -> [toc][heading]{title}
% \chapter[memoir 클래스에서 한글을 쓰자]{memoir 클래스에서\\ 한글을
%   쓰자} % -> [toc,heading]{title} %% 이 부분이 memoir와 다르다.
% \chapter{memoir 클래스에서 한글을 쓰자} % {toc,heading,title}
% \chapter[memoir 클래스에서 한글을 쓰자][memoir 클래스에서
% 한글...]{memoir 클래스에서\\ 한글을 쓰자}
% \chapter[memoir 클래스에서 한글...]{memoir 클래스에서\\ 한글을 쓰자}
% 주의: titleref에 \\ 문자가 들어가면 안된다. titleref에서 사용하는
% 것은 두번째 옵션인자이므로, 다음과 같이 두번째 옵션 인자를 밝혀줄
% 것.
\chapter[memoir 클래스에서 한글...][memoir 클래스에서 한글을 쓰자]{%
	memoir 클래스에서\\ 한글을 쓰자}
\traceon\label{sec:firstchap}\traceoff

\chapterprecis{\noindent 이 장에서는 memoir 클래스에 대해 간략히
	소개하고 한글화에 대하여 개관한다.}

\showcommand{chapterprecis}\showcommand{chapterstyle}\showcommand{pagestyle}
\thisclass\는 Peter Wilson 씨가 작성한 \LaTeX\ 클래스이다. \LaTeX 이 제공하는
표준 클래스\index{표준~클래스|bfhypidx}는 \cls{book}, \cls{report}, \cls{article}, \cls{letter}
등이 있지만, 어딘가 모르게 세세한 부분에서 부족한 점이 있어서 수많은 추가 패키지를
사용해야 원하는 문서 모양을 구현할 수 있는 경우가 많았다. \thisclass\는 그 동안 개발된
문서 조판의 세세한 부분을 하나의 클래스로 통합한 것으로, 사용자 입장에서는
정말 획기적인 환상적인 클래스가 아닐 수 없다. 매우 많은 패키지들을 이
클래스는 통합\cntrdot 내장하거나 그와 유사한 기능을 제공한다.
이 패키지의 결점은, 한번 사용하기 시작하면 다시는 book과 같은
표준 클래스로 돌아가기 어렵다는 점이다.

EUC-KR\index{한글!EUC-KR|bfhypidx} 한글을 \thisclass 에서 사용하도록 만들었던 것이 
\pkg{memhangul}였다.
이제 여기서 제공하는 클래스는 본질적으로 \pkg{memhangul}\과 동일하지만
한글을 UTF-8 유니코드로 입력할 수 있게 만든 것이다. \wi[유니코드]{유니코드 한글}을
처리하기 위해서 \pkg{dhucs}\를 채택하였다.
\query[유니코드]{유니코드 한글 처리}

\alineskip

\hangfrom{한글 }사용을 위해서 H\LaTeX 을 채택한 결과는 어느 정도
괜찮은 결과를 가져왔다. 그러나 \pkg{dhucs}\를 이용하여 구현한 \pkg{memhangul-ucs}\는
괜찮은 정도가 아니라 아주 훌륭한 결과를 얻게 되었다.

\showcommand{hangpara}\showcommand{hangfrom}%
\hangpara{2.6em}{2}%
{\SetAdhocFonts{unpg}{ungt}\showcommand{SetAdhocFonts}%
유니코드 한글 입력이 가능하도록 하는 것은 \pkg{dhucs}에서였다. 이것은
\LaTeX-ucs 패키지와 한글 자동조사 및 한글 문서서식을 합친 것인데,
핵심적인 한글 식자와 자동조사의 구현은 김도현 교수가 작성하였다.
필자는 사용자 인터페이스를 조금 추가하고 한글 문서서식을 보충하는
정도로 미미한 기여를 하였다.}

이 패키지의 장점은 대강 다음과 같다.
\tightlists\showcommand{tightlists}
\begin{itemize}

\item 유니코드를 쓴다는 것 자체가 장점이다. EUC-KR 한글의 범위를
넘어서서 맞춤법에는 어긋나지만 꼭 써야할 경우가 없지 않은 ``띡''과 같은
완성형 밖 글자를 식자할 수 있다.\footnote{%
    중세 한글 문제는 코드와 폰트의 문제가 얽혀 있어서 
    여기에서는 다루지 않는다.}

\item 절에 \wi{한글식 절카운터 모양}(section counter format), \dispcmd{pgana} 등을
사용하는 것이 쉽다.

\item \wi{자동조사} 기능이 구현되어 있다.

\item 주요 pagestyle과 chapterstyle이 \wi{한글}과 호환되게 하였다. 이호재 님의
말씀에 의하면, \thisclass 는 여러 가지 면에서 매우 편리하다고 한다.

\item 인덱스 만들기가 구현되었다.

\item \thisclass 의 여러 기능을 그대로 쓸 수 있다.

\end{itemize}

\section{이 문서의 컴파일 방법}

다음과 같은 순서로 컴파일한다.

\bvtopandtail\showcommand{bvtopandtail}\showenv{boxedverbatim}
\begin{boxedverbatim}
#> latex memucstest
#> latex memucstest
#> makeindex-dhucs -s dhucs memucstest
#> latex memucstest
\end{boxedverbatim}

|makeindex(-dhucs)|에 앞서서 |latex|을 두 번 실행하는 이유는,
memoir 패키지의 인덱스 만들기의 특징 때문이다. 처음 한 번만 실행해서는
|.idx| 파일이 만들어지지 않는다.

\section{chapter 스타일}

\showcommand{prechapternum}\showcommand{postchapternum}
사용자가 자신의 |chapterstyle|을 정의(定義)할 때는 반드시
\dispcmd{prechapternum}과 \dispcmd{postchapternum}을 함께 정의해주도록 한다.
그림~\ref{fig:examchapstyle}\은 |hangnum| chapter style을
정의하는 방법을 보여준다.
\begin{figure}
%% 웬만한 환경은 memoir에 다 있다.!!!
\begin{boxedverbatim}
\makechapterstyle{hangnum}{%
  \renewcommand{\chapnumfont}{\chaptitlefont}
  \settowidth{\chapindent}{\chapnumfont 999}
%  \renewcommand{\printchaptername}{} % <= 쓰지 않음.
  \renewcommand{\prechapternum}{} % <= 이 행을 정의
  \renewcommand{\chapternamenum}{}
  \renewcommand{\postchapternum}{}%  <= 이 행을 정의
  \renewcommand{\printchapternum}{%
    \noindent\llap{\makebox[\chapindent][l]{\chapnumfont 
            \thechapter}}}
  \renewcommand{\afterchapternum}{}
}
\end{boxedverbatim}
\caption{chapterstyle 정의 예제}\label{fig:examchapstyle}
\end{figure}

장 스타일을 수정하는 구체적인 예는 |memman.tex|\cite{memman}에서
가져온 |veelo| chapter style의 정의를 참조하라. 이 문서의
소스 Preamble에 있다. 예제는 \Cref{sec:math} \titleref{sec:math}\을 보라.
%% disable \nameref
\textbf{\nameref{sec:appchap}}\를 보라.

\textbf{\titleref{sec:appchap}}\를 보라. \pref{sec:appchap}\을, 
\ref{sec:appchap}\을, \Cref{sec:appchap}\을
\Sref{sec:appsec}\가, \ref{sec:appsec}\가, \titleref{sec:appsec}\가.
\titleref{sec:appsec}\을.
\showcommand{Cref}\showcommand{ref}\showcommand{pageref}\showcommand{titleref}
%% \Cref 명령으로 식자되는 결과를 주의깊게 보라.

\section{시집}\label{sec:poembook}

\wi{어머님}이 \wi{수술} 후에 거동이 불편해지신 후, 집에만
계시는 것이 무척 \wi{무료}하신 듯하다. 예전에 내가 보던
\wi{시집}을 꺼내 보고 계시다. \showcommand{titleref}\showcommand{Sref}%
\Sref{sec:poembook} \textit{\titleref{sec:poembook}}\는
텍스트 입력의 예제이다.

%% memoir는 poem 스타일을 자체 내장하고 있다.
%% plain poem title을 선언함.
\PlainPoemTitle\showcommand{PoemTitle}\showcommand{PlainPoemTitle}
\showenv{snugshade}\showcommand{poemtitle}\showenv{verse}
\begin{snugshade}
\renewcommand\poemtoc{subsection}
\settocdepth{subsection}
\poemtitle{신안리에서}
\settowidth{\versewidth}{\hbox{사람들은 굳이 한 마디 말 하려 하지 않는다.}}
\ifpdf\begin{verse}\else\begin{verse}[\versewidth]\fi

실개천 너머 나 있는 샛길로\\
택시가 들어오고 나가며 \\
겨르로운 달빛에 감출 부끄러움도 없는\\
사람들은 굳이 한 마디 말 하려 하지 않는다.\\!

나는 톱밥 같은 달빛을 한 웅큼\\
멀리 철길 쪽으로 뿌렸다.\\
매달린 불빛보다\\
아침이 먼저 깨어나는 광경을 보며.\\
삶은 새의 낮은 날음새 같은 것,\\
이대로 누워 잠들기\\
두렵지 않은 작은 마을에\\
더 작은 풀꽃 인사한다.\\
잊혀짐 너머 안부 묻는다.\\
\end{verse}
\end{snugshade}

이 스타일은 \pkg{memucs-setspace}\를 이용한다. 이 패키지는\footnote{
    이 스타일은 \pkg{setspace}\를 수정한 것이다.
    \env{verse}, \env{quote}\와 같은 환경을 조금 바꾸어서 행간을 약간 줄여주는
    기능이 있다.}
verse, quote와 같은 환경을 조금 바꾸어서 행간을 약간 줄여주는
기능이 있다. 이 부분에서 \dispcmd{adjustfloatfnspacing}을 불러보겠다.
\showcommand{adjustfloatfnspacing}
다음번 각주는 행간이 조금 달라져야 한다.
\adjustfloatfnspacing
float 안에 놓인 것과 같아지도록.\footnote{%
이 스타일은 \pkg{setspace}\를 수정한 것이다.
verse, quote와 같은 환경을 조금 바꾸어서 행간을 약간 줄여주는
기능이 있다.}

%%%\showcommand{memucsfninterwordhook}\showcommand{xspaceskip}
%%%만약 각주의 자간을 조절하고 싶으면 어떻게 할 것인가?
%%%이 때는 \dispcmd{memucsfninterwordhook}이라는 명령을 재정의해준다.
%%%이것을 간편하게 재정의하는 명령을 따로 제공하지 않는 이유는,
%%%이것을 재정의하는 것이 그다지 바람직하지 않기 때문이다.
%%%게다가 만약 이 명령에서 \dispcmd{xspaceskip} 등을 수정하면
%%%본문에도 영향을 미치기 때문에, 본문 중에서는 
%%%되도록 사용하지 않는 것이 좋다. 여기서는 오직 예시만을
%%%위하여 간단한 예를 들어두겠다.
%%%\let\ORIGmemucsfninterwordhook\memucsfninterwordhook
%%%\renewcommand\memucsfninterwordhook{%
%%%    \ifLuaTeX
%%%    \hangulfontspec[InterHangul=.5em]{\DefaultFont}
%%%    \else\ifXeTeX
%%%    \hangulfontspec[interhchar=.5em]{\DefaultFont}
%%%    \else\ifPDFTeX
%%%    \interhchar{.5em}
%%%    \fi\fi\fi
%%%    \spaceskip=.8em plus .133em minus .111em
%%%%   \xspaceskip=1em plus .05em minus .05em 
%%%}%
%%%이제 각주를 붙여보자.\footnote{%
%%%각주의 자간을 수정하려면
%%%위와 같이 한다. 잘 되는가?}
%%%
%%%\let\memucsfninterwordhook=\ORIGmemucsfninterwordhook

%\DefaultInterHchar

\showcommand{noadjustquotespacing}\showcommand{adjustquotespacing}
이 아래는 본문 중에서 \dispcmd{noadjustquotespacing}과 \dispcmd{adjustquotespacing}을
불렀을 때 \env{quote} 안에서 행간이 어떻게 변하는지 보여준다.
기본값은 \dispcmd{noadjustquotespacing}이다.

\begin{quote}
이 스타일은 \pkg{setspace}\를 수정한 것이다.
verse, quote와 같은 환경을 조금 바꾸어서 행간을 약간 줄여주는
기능이 있다. 마치 float 안에 놓인 것과 같이.
\end{quote}

여기서 \dispcmd{adjustquotespacing}을 호출함.

\adjustquotespacing
\begin{quote}
이 스타일은 \pkg{setspace}\를 이용한다. 이 패키지는
verse, quote와 같은 환경을 조금 바꾸어서 행간을 약간 줄여주는
기능이 있다. 마치 float 안에 놓인 것과 같이.
\end{quote}

verse류 환경의 행간은 quote를 따른다. 앞서 보인 시를 여기에서
다시 식자해보자.
\begin{snugshade}
\renewcommand\poemtoc{subsection}
\settocdepth{subsection}
\PoemTitle{신안리에서}
\settowidth\versewidth{사람들은 굳이 한 마디 말 하려 하지 않는다.}
\ifpdf\begin{verse}\else\begin{verse}[\versewidth]\fi

실개천 너머 나 있는 샛길로\\
택시가 들어오고 나가며 \\
겨르로운 달빛에 감출 부끄러움도 없는\\
사람들은 굳이 한 마디 말 하려 하지 않는다.\\!

나는 톱밥 같은 달빛을 한 웅큼\\
멀리 철길 쪽으로 뿌렸다.\\
매달린 불빛보다\\
아침이 먼저 깨어나는 광경을 보며.\\
삶은 새의 낮은 날음새 같은 것,\\
이대로 누워 잠들기\\
두렵지 않은 작은 마을에\\
더 작은 풀꽃 인사한다.\\
잊혀짐 너머 안부 묻는다.\\
\end{verse}
\end{snugshade}

여기서 다시 \dispcmd{noadjustquotespacing}을 선언함.

\noadjustquotespacing
\begin{quote}
이 스타일은 \pkg{setspace}\를 이용한다. 이 패키지는
verse, quote와 같은 환경을 조금 바꾸어서 행간을 약간 줄여주는
기능이 있다. 마치 float 안에 놓인 것과 같이.
\end{quote}

\section{강조}

강조를 구현하는 데는 여러 가지 방법이 있다. 서구 문헌의
경우 이탤릭체를 사용하는 것이 일반적이고, 예전의 독일 문헌에서는
자간을 띄우는 강조 방법을 사용한 적도 있다고 한다.

우리말 문헌은 일관되어 있지 않다. 

\subsection{방점 강조}

예를 들면 \circemph{방점 강조} 방법을 사용하는 것이 가능하다.
\wi[강조]{방점 강조}는 \dotemph{한글 맞춤법}의 문장부호 조항에서 규정하고 있는 방법이기도
하다.\showcommand{circemph}\showcommand{dotemph}
맨처음에 이 명령은 H\LaTeX\ 1.01에서 구현된 것이었는데,
그 후 
\useremph{발전을 거듭하여}
현재에 이르렀다.
\showcommand{useremph}

\subsection{기울인 글꼴 또는 글꼴 대체}

\ifPDFTeX \ungremph \fi
H\LaTeX{}에서는 {\itshape\wi[강조]{기울인 글꼴}}을 쓰는 방법을 오랫동안 사용해왔다.
그러나 실제로 출판되는 서적에서는 \wi[강조]{글꼴 대체} 방법을 사용하는
경우가 많다.

\ifPDFTeX \begingroup
	\regremph
	|gremph| 옵션은 다음과 같은 효과를 낸다. \emph{gremph 옵션}.
	\ungremph
	이번에는 이것을 disable한다. \emph{gremph 옵션}. 
	
	|\bfemtrue|와 |\bfemfalse|는 그래픽 글꼴 대신
	은바탕 굵은 글꼴로 식자하게 한다. 또, |\SetGremphFonts|
	명령으로 직접 폰트를 지정할 수 있다.\showcommand{regremph}		\showcommand{ungremph}
	\showcommand{SetGremphFonts}\showcommand{bfemtrue, \textbackslash bfemfalse}
	다만 |\SetGremphFonts| 명령은 preamble에서만 쓸 수 있다.
\endgroup \else\ifLuaOrXeTeX \begingroup
	\hologo{LuaLaTeX}이나 \hologo{XeLaTeX}을 쓰는 경우,
	강조 글꼴의 선택은 폰트 속성과 연계된다. 즉, \textit{Italic Font}로
	지정된 글꼴이 강조 글꼴로 사용된다. 
	|[itemph]| 옵션은 이 부분의 글꼴을 기울어지게 만든다.
\endgroup \fi\fi

\showcommand{MakeShortVerb}

%% 한글 섹션 카운터 포맷을 가능하게 함.
\renewcommand\thesection{\pgana{section}}
\section{두번째 절}

\epigraphtextposition{flushleftright}
\epigraph{%
나는 십대에 \wi{철학책}을 읽기 시작한 무렵부터
거기에 언제나 이 `나(私)'가 빠져 있다고 느껴왔다.}
{\emph{탐구}\\ \textsf{\wi[인명]{카라타니 코진(柄谷行人)}}}

\showcommand{pgana}\showcommand{epigraph}
\showcommand{epigraphtextposition}

\wi{수학}에서는 어떤 공리계가 하나의 해석 모델에서는 참이지만
다른 해석 모델에서는 거짓인 일이 있을 수 있다.
이것은 \wi{공리계}가 불충분한 경우이다.

\renewcommand\thesection{\onum{section}}
\section{표}

\showcommand{legend}\showcommand{onum}
\thisclass\는 다양한 표작성 환경을 제공한다.
\tref{tab:test}\과 \tref{tab:test2}\를 보라.
\showcommand{tref}

%% tabular에는 legend를 붙일 수 있다.
\begin{table}
\caption{test table}\label{tab:test}
\centering
\begin{tabular}{cc}
\hline
두 줄 짜리 & 3 \\
하단 제목 붙은 & 4 \\
\hline
\end{tabular}
\legend{caption 아닌 소제목}
\end{table}

\begin{table}
\caption{test table 2}\label{tab:test2}
\centering
\begin{tabular}{cc}
\hline \hline
두 줄 짜리 & 6 \\ \hline
하단 제목 안 붙은 & 5 \\
\hline \hline
\end{tabular}
\end{table}

(여기서 페이지를 나눈다. 페이지 나누기 명령으로는
\dispcmd{cleartoverso}를 사용하겠다. 이것은 이 뒤에 새로
시작하는 페이지가 짝수쪽(verso)이 되게 한다. \dispcmd{cleartorecto} 
명령도 있다.)
\showcommand{cleartoverso}\cleartoverso

\namesubappendixtrue
\begin{subappendices}
\addappheadtotoc
\section{자동조사 테스트}
\showcommand{namesubappendixtrue}
\showcommand{addappheadtotoc}
\showenv{subsppendices}

간단한 \dispcmd{ref}-like 명령으로 \wi{자동조사}를 테스트한다.
\dispcmd{cite}도 잘 된다.

\tref{tab:test}\와 \tref{tab:test2}\이 어떻게 보이나요?
그리고 \fref{fig:examchapstyle}\은 \pref{fig:examchapstyle}\로
가면 볼 수 있어요. 페이지~\pref{fig:examchapstyle}\로 가보세요.
\showcommand{tref, \textbackslash fref}\showcommand{pref}

위의 문장 입력:
\bvsides\showcommand{bvsides}\showenv{boxedverbatim}
\begin{boxedverbatim}
\tref{tab:test}\와 \tref{tab:test2}\이 어떻게 보이나요?
그리고 \fref{fig:examchapstyle}\은
\pageref{fig:examchapstyle}\AltPageName\로 가면 볼 수 있어요.
페이지~\pageref{fig:examchapstyle}\로 가보세요.
\end{boxedverbatim}

\section{그림 테스트}

\newlength{\mylength}
\begin{figure}
\calccentering{\mylength}
\begin{adjustwidth*}{\mylength}{-\mylength}
\centering
\includegraphics[width=.6\textwidth]{doeun}
\caption{공부하는(?) 도은이}\label{doeunbike}
\end{adjustwidth*}
\end{figure}

그림이 잘 들어가는지도 테스트해야 한다고 한다. \fref{doeunbike}\를 볼 것.
이 그림은 마진폭을 계산해서 편집영역의 중간이 아니라 페이지 전체의
중간으로 가도록 \dispcmd{calccentering}을 이용했다.\showcommand{calccentering}
\showenv{adjustwidth*}
이 계산이 표준 \LaTeX\ 클래스에서 얼마나 귀찮은 것이었는지 상상할 수 
있겠는가?
\query{그림넣기}

\begin{boxedverbatim}
\newlength{\mylength}
\begin{figure}
\calccentering{\mylength}
\begin{adjustwidth*}{\mylength}{-\mylength}
\centering
\includegraphics[width=.6\textwidth]{doeun}
\caption{공부하는(?) 도은이}\label{doeunbike}
\end{adjustwidth*}
\end{figure}
\end{boxedverbatim}

\end{subappendices}

%%% part 페이지를 깨끗하게...
\copypagestyle{part}{empty}

%%%%%%%%%% Ruled를 사용하되 partmark를 정의함.
\copypagestyle{MyRuled}{Ruled}
\newlength{\MyRuledheadwidth}
\setlength{\MyRuledheadwidth}{\textwidth}
\addtolength{\MyRuledheadwidth}{\marginparsep}
\addtolength{\MyRuledheadwidth}{\marginparwidth}
\makerunningwidth{MyRuled}{\MyRuledheadwidth}
\makeheadrule{MyRuled}{\MyRuledheadwidth}{\normalrulethickness}
\makeheadposition{MyRuled}{flushright}{flushleft}{flushright}{flushleft}
\makeatletter
\makepsmarks{MyRuled}{%
  \let\@mkboth\markboth
  \def\partmark##1{\markboth{\hparttitlehead. ##1}{##1}}
  \def\chaptermark##1{\markright{\hchaptertitlehead. ##1}}
  \def\sectionmark##1{}
}
\def\partmark#1{\markboth{\hparttitlehead. #1}{#1}}
\makeatother

%%% 제2장. 페이지 스타일과 챕터 스타일을 바꿈.
%%% 새로운 장에 새로운 스타일을 적용하려 할 때는,
%%% chapterstyle은 \chapter명령보다 먼저 부르고 \pagestyle은
%%% \chapter보다 나중에 부르는 것이 좋다.

\part{테스트는 즐거워}
\oblivoirchapterstyle{veelo}
\pagestyle{MyRuled}

\chapter{수학질문상자}\traceon\label{sec:math}\traceoff

%% 절의 카운터를 \pnum으로.
\renewcommand\thesection{\pnum{section}}

\showcommand{makepagestyle}\showcommand{copypagestyle}

\section{자연로그의 밑}

이 절은 \cite{Kentaro}\을 인용하였다. \env{singlespacing}\을
사용하였다.\showenv{singlespacing}

%\begin{singlespacing}
\[
e^x =1+ \frac{x}{1!} + \frac{x^2}{2!} + \frac{x^3}{3!} + \cdots
\]
이라는 사실이 알려져 있다. 여기에서 $x=1$이라 하면,
\[
e=1+\frac{1}{1!}+\frac{1}{2!}+\frac{1}{3!} +\cdots
\]
가 된다.

우선
\[
\lim_{n\to\infty}\left(1+ \frac{1}{n}\right)^{n} = e
\]
에서 $1/n = h$라 두면,
\[
\lim_{h\to 0}\left(1+h\right)^{\frac{1}{h}} = e
\]
라고 쓸 수 있다. $e$를 밑으로 하는 대수를 $\log$라고 표시하면,
\[
\frac{\log(1+h)}{h} = \log(1+h)^{\frac{1}{h}}
\]
인데, 여기에서 $h \to \infty$이라면,
\[
\lim_{h\to 0}\frac{\log(1+h)}{h} = \lim_{h\to 0}\log(1+h)^{\frac{1}{h}} =\log e.
\]

따라서,
\[
\log(1+h)=x, \quad\text{즉}\quad h=e^{x} -1
\]
이다. 여기에서,
\[
1=\lim_{h\to0}\frac{\log(1+h)}{h}=\lim_{x\to0}\frac{x}{e^x -1}.
\]

따라서,
\[
\lim_{x\to0}\frac{e^x -1}{x}=1
\]
을 얻는다. 그런데 여기에서,
\[
y=e^x
\]
의 도함수 $y'$를 구해본다.
\[
y'=\lim_{h\to0}\frac{e^{x+h}-e^x}{h}=e^x \lim_{h\to0}\frac{e^h -1}{h}=e^x.
\]
따라서,
\[
y=e^x \text{라면}, \qquad y'=e^x
\]
이다.

또, $y=\log x$의 도함수를 구해본다.
\begin{displaymath}
\begin{split}
y' &= \lim_{h\to0}\frac{\log(x+h)-\log x}{h} \\
   &= \lim_{h\to0}\frac{1}{h}\log\left(1+ \frac{h}{x}\right) \\
   &= \frac{1}{x}\lim_{h\to0}\frac{x}{h}\log\left(1+\frac{h}{x}\right) \\
   &= \frac{1}{x}\lim_{\frac{h}{x}\to0}\log\left(1+\frac{h}{x}\right)^{\frac{x}{h}} \\
   &= \frac{1}{x}\log e \\
   &= \frac{1}{x}.
\end{split}
\end{displaymath}

따라서, 
\[
y=\log x \text{라면}, \qquad y'=\frac{1}{x}.
\]

%\end{singlespacing}

이와 같이 간단한 공식으로 얻어진 것은 대수의 밑으로 $e$를 썼기
때문이다. $e$ 이외의 밑을 사용하면, 공식은 보다 복잡하게 된다.
이런 의미에서 $e$를 밑으로 하는 대수를 \emph{자연로그}라고 부른다.\showcommand{emph}
이상으로부터 상상할 수 있듯이, 미적분학과 같은 이론을 전개할 때는
$e$를 밑으로 하는 대수를 사용하고, 실제의 수치계산에서는
$10$을 밑으로 하는 상용로그를 사용한다.

\fancybreak{* * *}

$0 \le t \le 1$에 있어서, $f(t)$는 연속인 도함수 $f'(t)$를 가지고,\showcommand{fancybreak}
$0 < f'(t) \le 1,\quad f(0)=0$이다. 이 때 다음 부등식이 성립함을 보여라.
\[
\left[ \int_{0}^{1} f(t)dt \right ] ^2 \ge \int_{0}^{1} [f(t)]^3 dt 
\]

\vskip\onelineskip

적분구간의 상한을 변수 $x$로 바꾸어본다.
\[
\left[ \int_{0}^{x} f(t)dt \right]^2 \ge \int_0^x [f(t)]^3 dt
\]
이 식의 좌변에서 우변을 빼서 그것을 $F(x)$라고 두자.
\[
F(x) = \left[ \int_{0}^{x} f(t)dt \right]^2 - \int_0^x [f(t)]^3 dt
\]
이 때 $F(0)=0$이다.

$F(x)$를 $x$로 미분하면
\begin{equation}
\begin{split}
F'(x) &= 2\left[ \int_{0}^{x} f(t)dt \right]f(x) - \{f(x)\}^3 \\
  &= f(x) \left[ 2\int_{0}^{2} f(t)dt - \{f(x)\}^2 \right]
\end{split}
\end{equation}

문제의 의미에 의해 $0<x<1$에서 $f(x) > 0$이다. 
\[
G(x) = 2 \int_{0}^{x} f(t)dt - \{f(x)\}^2 \qquad (0 \le x \le 1)
\]
이라 놓으면 $G(0)=0$이고,
\begin{equation}
\begin{split}
G'(x) &= 2f(x) - 2f(x)f'(x) \\
      &= 2f(x) \{1-f'(x)\} \ge 0
\end{split}
\end{equation}
이다. $0 \le x \le 1$인 모든 $x$에 관하여 $f'(x) \le 1$이므로
$G'(x) \ge 0$. 따라서 $G(x) \ge 0$임을 말할 수 있다. 그러므로
\uline{$F(x) \ge 0$이 성립한다.}\showcommand{uline (ulem package)}

%% 부록 관련 명령
%% \appendix 또는 appendices 환경
%\appendix
%% 부록 면주에 \hparttitlehead, \hchaptertitlehead 를 표시하지 않는다.
\def\partmark#1{\markboth{#1}{#1}}
\def\chaptermark#1{\markright{#1}}
%

\AppendixTitleToToc
\AttachAppendixTitleToSecnum

%\begin{appendices}
\appendix
\appendixpage*

\renewcommand\thechapter{\Roman{APPchapter}}
\renewcommand\thesubsection{\thesection.\arabic{APPsubsection}}
\setcounter{APPchapter}{0}

\chapterstyle{appendixdefault}
\renewcommand*\prechapternum{\chapnamefont 부록\ \ 제}
\renewcommand*\postchapternum{\chapnamefont 장}
\renewcommand*\printchapternum{\chapnumfont\thechapter}
%% appendix에서는 chaptersyle을 appendixcompanion,
%% appendixdefault, appendixsection 등으로 지정할 것.
%% 사용자가 새로운 chapterstyle을 설정하려 할 때는
%% appendixXXXX 환경을 새로 만들어야 한다.
%% appendix에서의 절 모양은 \thechapter.\arabic{section}으로
%% 된다. 이것은 renewcommand할 수 있다.

\pagestyle{hangul}

\chapter{19세기 초중엽}\label{sec:appchap}

\chapterprecis{\noindent 부록에서는 한자가 많은 문장과 상호참조가 많은 문장을
	시험한다.}
\showcommand{chapterprecis}

\showcommand{appendix}\showenv{appendices}
\ResetHangulspace{1.333}{1.2}
\showcommand{ResetHangulspace}
\paragraphfootnotes
\showcommand{paragraphfootnotes}

\section{평안도 광산}\label{sec:appsec}
19세기에 들어와서도 1807년 \wi{평안도} \wi[위원]{渭原} 지방에서
\wi{광산}이 개발되자 단시일내에 수많은 광산노동자가 집결하여 커다란
광산촌을 형성하였는데 이 광산도 앞의 遂安광산과 동일한 형태로
운영되었을 것은 틀림없다. 1811년 평안도 농민전쟁 당시 서울에 있던
禹君則의 物主(資本主)가 수천 냥의 자금을 보내어 그것으로 雲山의 금광을
운영하게끔 한 사실, 또한 앞서 본 대로 雲山 광산노동자 800명을 官軍으로
강제적으로 끌어들이려 한 계획, \wi[김창시]{金昌始}가
\wi[운산광산]{雲山광산}의 \wi[우욱]{禹郁}과 연계하에 그곳의
광산노동자를 농민전쟁에 끌어들이는 조직 사업을 진행하였다고 하는
사실은\footnote{\bnm{日省錄} 哲宗 9年 2月 3日條} 이 시기에
雲山금광에서도 수많은 광산노동자가 집결하여 있었음을 確證하여
준다. 또한 1858년 함경도 암행어사 洪承裕의 보고에 의하면 당시
함경도에서는 金, 銀, 銅의 潛採가 광범하게 이루어져서 하나의 광산이
開發되기만 하면 도처에서 金店軍이 몇천 명씩 몰려와서 鑛山村이 형성되고
場市가 열려서 각종 日用品이 광범하게 매매되고 있고 광산의 坑의 깊이가
千餘尺에 달하는 경우도 적지 않았다.\footnote{\bnm{關南平亂錄} 卷五
  安州牧使 牒報條.}

\section{술어 논리의 완전성 정리}

이 절의 텍스트는 \cite{incompl}\ pp.\ 155ff.에서 취하였다.
\showcommand{cite}

\subsection{먼저 기호의 설명을 잠깐\texorpdfstring{\ldots}{...}}

술어논리의 형식화는 모든 대상과 논리법칙(공리)의 기호화로부터
시작된다. ``태초에 기호가 있다''이다.

\showcommand{texorpdfstring}
기본 기호는 \fref{fig:syms}의 여섯 종류이다.
\showcommand{fref}

\begin{figure}
\centering
\begin{minipage}{.75\textwidth}
\begin{tabbing}
1111\=11111111111111111111\= \kill
(1) \> 대상기호(상수) \> $ c_{1}, c_{2}, c_{3}, c_{4}, \ldots $ \\
(2) \> 함수기호 \> $ f_{1}, f_{2}, f_{3}, f_{4}, \ldots $ \\
(3) \> 술어기호 \> $ P_{1}, P_{2}, P_{3}, P_{4}, \ldots $ \\
(4) \> 자유변수 \> $ a_{1}, a_{2}, a_{3}, a_{4}, \ldots $ \\
(5) \> 속박변수 \> $ x_{1}, x_{2}, x_{3}, x_{4}, \ldots $ \\
(6) \> 논리기호 \> $ \land, \lor, \rightarrow, \lnot, \forall, \exists $
\end{tabbing}
\end{minipage}
\caption{기본 기호}\label{fig:syms}
\end{figure}

\showenv{tabbing}\showenv{minipage}

이 중, `자유변수'란 불특정의 대상, `속박변수'란 논리기호 $\forall$과
$\exists$의 어느 것인가와 함께 사용하는 변수를 나타낸다. $\forall$과
$\exists$는 `속박기호' 또는 `양화기호'라 부르고 $\forall$을
`전칭기호', $\exists$를 `존재기호'라고 말한다.

이 $\forall$과 $\exists$를 포함하지 않는 논리 체계가 \bnm{프린키피아
  마테마티카}에서 처음으로 체계화된 ``명제논리학''이다. 이쪽은 벌써
1920년에 그 무모순성과 완전성이 당시 컬럼비아 대학의 학생에 지나지
않았던 E.~L.~포스트에 의해서 증명되고 있다.

논리기호는 초수학에 있어서는 추상적인 기호에 지나지 않는다. 그러나
근원을 밝히면 그것들에는 원래 각각 고유한 의미가 있다. 즉 `$\lnot$'는
``\ldots 이 아니다'', `$\lor$'는 ``또는'', `$\land$'는 ``동시에'',
`$\rightarrow$'는 ``이라면'', $\forall$은 ``모든'', $\exists$는 ``어떤
\ldots 가 존재한다''는 의미이다.

이 해석으로 말하면 대상이 유한집합의 경우는 술어논리도 명제논리로
환원될 수 있다. $\forall$과 $\exists$는 무한집합을 대상으로 할 때
비로소 의미를 갖는 논리기호이다. 수학은 본질적으로 무한집합을 대상으로
하고 있기 때문에 술어논리는 수학의 논리로 되어 있는 것이다.

그러나 지금은 이러한 유래를 제외하고 형식적 체계만을 문제삼고 있기
때문에 당분간 기호의 의미는 잊어버려도 상관없다. 오히려 적극적으로
잊어버려서 기호의 조작에만 전념하려고 하는 것이 초수학의 기본
방침이다.

\subsection{`항', `논리식'의 정의와 논리법칙}

기호가 갖추어진 곳에서 추론의 대상이 되는 `항'을 \pref{fig:term}의
\fref{fig:term}\과 같이 정의한다.\showcommand{pref}

\begin{figure}
\centering
\begin{minipage}{.75\textwidth}
\begin{enumerate}[(1)]\tightlist\small\raggedright
\item \label{itemone} 대상 기호와 자유변수는 항이다.
\item \label{itemtwo} $f$가 $n$ 변수의 함수기호이고, $t_1 , t_2 ,
  \ldots, t_n$이 항이라면 $f(t_1 , t_2 , \ldots t_n )$은 항이다.
\item (\ref{itemone})\와 (\ref{itemtwo})\으로부터 얻어지는 것만이 항이다.
\end{enumerate}
\end{minipage}
\caption{`항'의 정의}\label{fig:term}
\end{figure}

다음으로 이 항을 사용해서 `논리식'을 \fref{fig:logi}\과 같이 정의한다.

\begin{figure}
\centering
\begin{minipage}{.85\textwidth}
\begin{enumerate}[(1)]\tightlist\small\raggedright
\item \label{st} $P$가 $n$ 변수의 술어기호이고, $t_1 , t_2 , \ldots ,
  t_n $이 항이라면 $P(t_1 , t_2 , \ldots , t_n )$은 논리식이다. 특히
  이것을 \emph{원시논리식}이라 부른다.
\item \label{nd} $A, B$가 논리식일 때 $\lnot A, A \lor B, A \land B, A
  \to B$도 논리식이다.
\item \label{rd} $A(a)$가 자유변수 $a$를 포함하는 논리식이고 $x$가
  $A(a)$ 속에 나타나지 않는 속박변수일 때 $\forall xA(x), \exists
  xA(x)$는 논리식이다.
\item (\ref{st}), (\ref{nd}), (\ref{rd})에 의해서 얻어지는 것만이
  논리식이다.
\end{enumerate}
\small
덧붙여 말하면, $\forall xA(x)$는 ``모든 $x$는 $A$를 충족시킨다'',
$\exists xA(x)$는 ``$A$를 충족시키는 $x$가 존재한다''라고 해석한다.
\end{minipage}
\caption{`논리식'의 정의}\label{fig:logi}
\end{figure}

그러면 다음은 이들 논리식을 사용해서 추론을 행하기 위한 논리법칙의
설정인데, 힐베르트-아카만의 공리계에서는 다음 \fref{fig:rules}\과 같이
되어 있다.

\begin{figure}
\centering
\begin{minipage}{.85\textwidth}\small
\noindent\textsf{공리}\\
\begin{enumerate}[(1)]\tightlist
\item $A \to (B\to A)$
\item $(A\to B)\to ((A\to (B\to C))\to (A\to C))$
\item $A\to (B\to A\land B)$
\item $A\land B\to A,\quad A\land B\to B$
\item $A\to A\lor B,\quad B\to A\lor B$
\item $(A\to C)\to ((B\to C)\to(A\lor B\to C))$
\item $(A\to B)\to ((A\to \lnot B)\to \lnot A))$
\item $\lnot\lnot A\to A$
\item $A(t)\to \exists xA(x)$ ($t$는 항)
\item $\forall xA(x) \to A(t)$ ($t$는 항)
\end{enumerate}

\bigskip

\noindent\textsf{추론 규칙}\\
\begin{enumerate}[1~~]
\item \label{rules:st} $\dfrac{A,\; A\to B}{B}$
\item $\dfrac{A(a)\to C}{\exists xA(x)\to C}$
\item $\dfrac{C\to A(a)}{C\to \forall xA(x)}$
\end{enumerate}
다만, $A, B, C, \ldots, \forall xA(x), \ldots$ 등은 모두 논리식으로
한다.
\end{minipage}
\caption{논리법칙 (힐베르트-아카만의 공리계를 따름)}\label{fig:rules}
\end{figure}

`추론규칙'이 형성하는 ``도형''을 ``연역도'' 또는 ``증명도''라 부르고
이 도형은 ``위의 기호열로부터 아래의 기호열을 추론한다''라고
해석한다. 예컨대 추론규칙 \ref{rules:st}\는 ``$A$와 $A\to B$로부터
$B$를 추론한다''라고 해석하는 것이다.

\subsection{증명과 해석}

이만큼 준비가 된 곳에서 `증명가능'의 개념을 다음과 같이 정의한다.\showcommand{tightlist}
\showenv{enumerate}
\begin{enumerate}[(1)]\tightlist
\item \label{ev:st} 공리는 증명가능하다.
\item \label{ev:nd} 증명가능한 논리식에 추론규칙을 적용해서 얻어지는 논리식은
  증명가능하다.
\item (\ref{ev:st})\과 (\ref{ev:nd})에서 얻어진 논리식만이
  증명가능하다.
\end{enumerate}

논리식 $A$가 $B_1 , B_2 , \ldots , B_n $을 가정했을 때 증명할 수
있다면,
\begin{displaymath}
B_1 , B_2 , \ldots , B_n \vdash A
\end{displaymath}
라고 쓴다.\showenv{displaymath}

또한 $A$가 술어논리의 공리만을 사용해서 증명가능할 때는
\begin{displaymath}
\vdash A
\end{displaymath}
라고 쓴다.

\begin{snugshade}
실례를 두 가지 대비시켜서 보여주자.

예컨대 논리식,
$$
\forall x \exists y (y < x)
$$
는 ``모든 $x$에 대해서 $y$가 존재하고 $y$는 $x$보다 작다''라고 해석할
수 있다. 이 해석 아래에서는 실수의 영역이면 모델이 되나 자연수의
영역에서는 모델이 되지 않는다.  자연수로 $x$를 $0$으로 잡으면 그것보다
작은 자연수 $y$는 존재하지 않기 때문이다.

대비적인 실례로서 논리식,
$$
\exists x \forall y (x \le y)
$$
를 취하자. 그 해석은 ``어떤 $x$가 존재하고 모든 $y$에 대해서 $y$는
$x$와 같거나 $x$보다 크다''로 된다.  이것은 자연수가 모델이라면
$0$이라는 최소수가 존재하기 때문에 진실이 되나 실수의 모델에서는 허위로
되는 논리식이다.
\end{snugshade}
\showenv{snugshade}

그래서 모든 해석에 대해서 진실이 되는 논리식을 `항진식' 또는
`토톨로지'라 부르고 $A$가 토톨로지일 때
\begin{displaymath}
\vDash A
\end{displaymath}
라고 쓰기로 한다.

\chapter{이 문서에 관하여}

이 문서는 \pkg{memhangul-ucs}를 처음 만들 때 테스트용으로 작성한
것이다.

\pkg{oblivoir}는 원래 \pkg{dhucs}를 백그라운드 드라이버로 하여
구현된 것이었다. 그러다가 \hologo{XeTeX}ko, \hologo{LuaTeX}ko를
백그라운드로 한 \pkg{xoblivoir}가 만들어졌다.
현재 버전의 \pkg{xoblivoir}는 이 둘을 합쳐서 어떤 엔진으로 컴파일해도
거의 같은 결과를 얻게끔 되어 있다.
따라서, 이 문서는 \hologo{pdfLaTeX}, \hologo{XeLaTeX}, \hologo{LuaLaTeX} 어느 것으로도 컴파일된다. \showcommand{hologo}

\section{kotexindy에 관하여}
찾아보기를 만들려면 |kotexindy|를 사용하라. 예를 들면 다음과 같다.
\begin{verbatim}
$ kotexindy memucstest.idx
\end{verbatim}
찾아보기를 찍는 명령은 |\printindex|이다. \showcommand{printindex}

%\restorechapsec \showcommand{restorechapsec}
%% 만약 appendix가 문서의 가장 끝에 오는 것이 아니라면
%% 이 명령을 appendix 이후에 실행해준다.
%% 아래와 같이 appendices 환경을 쓰는 경우에는
%% 환경을 종료하기만 하면 된다.
%\end{appendices}

%%% 본문의 끝.
\backmatter
\chapterstyle{demo}

%% \bibintoc 하면 참고문헌이 목록에 나온다.
%% 기본값이므로 별도로 설정할 필요는 없다.
%\bibintoc
\renewcommand\prebibhook{%
    \showcommand{bibintoc}%
    이 참고문헌 예제는 시험을 위해서 작성된 것으로 실제 문서의
    내용과는 무관하다.%
    \showcommand{prebibhook}}
\begin{thebibliography}{00}
\bibitem[카누쓰86]{Knuth} Donald Knuth. \textit{The
    \TeX{}book}. Addison-Wesley. 1986.
\bibitem{memman} Peter Wilson.  ``The Memoir Class for Configurable
  Typesetting --- User Guide,'' On-line 문서.
  \url{http://www.ctan.org/tex-archive/macros/latex/contrib/memoir/}.
\bibitem{incompl} 요시나가 요시마사, 임승원 (옮김), \bnm{불완전성 정리
    --- ``이성의 한계''의 발견}. 전파과학사. 1993.
\bibitem[수학질문상자]{Kentaro} 야노 겐타로(矢野健太郞), 전재복
  (옮김).  \bnm{수학 질문 상자 --- 왜일까? 그것을 알고
    싶다}. 전파과학사. 1991.  (원저: 고단샤, 1973).
\end{thebibliography}

\indexintoc 
\renewcommand\preindexhook{%
  찾아보기는 테스트를 위해서 임의의 단어들로 선정되었다.
  \bigskip}
\printindex

%% memoir에서는 \listof... 명령을 아무데나 선언할
%% 수 있다. 신기하다.
\clearpage
\showcommand{listof...}
\listofqueries

\end{document}

